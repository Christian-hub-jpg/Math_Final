\documentclass[mla8]{mla}
\usepackage{graphicx}
\title{Imaginary Coefficients}
\author{Christian Nunez}
\professor{Stanley Huddy}
\course{Differential Equations}
\date{12 December 2022}
\usepackage[backend=biber]{biblatex}
\addbibresource{Imaginary_Coeffiencts.bib}

\begin{document}
\begin{paper}

In the peer reviewed article \textit{The linear differential equations with complex constant coefficients and Schrödinger equations} by Soon-Mo Jung and Jaiok Roh they demonstrate a possible solution to a second-order inhomogeneous linear differential equations, $y^{\prime \prime}(x)+\alpha y^{\prime}(x)+\beta y(x)=r(x)$, with complex constant coefficients, by looking at some characteristics of approximate solutions to these kind of differential equations. As a practical example to this method that was applied to the time-independent Schrödinger's equations that is commonly used in quantum mechanics to describe the behavior of electrons when there is no observer.

In order to prove the existence of solution to a second-order inhomogeneous linear differential equation with complex constant coefficients, $y^{\prime \prime}(x)+\alpha y^{\prime}(x)+\beta y(x)=r(x)$, where $\alpha$ and $\beta$ are complex-valued constants and $\mathit{r}$ is a continuous function with a complex-valued output. Denoting $\lambda$ and $\mu$ are the roots to the characteristic equation, $x^2+\alpha x + \beta = 0$, where p = $\Re(\lambda)$ and q = $\Re(\mu)$. Let $\mathit{I}$ be any open interval between $-\infty$ and $\infty$.

\end{paper}
\printbibliography
\end{document}
%https://www.sciencedirect.com/science/article/pii/S089396591630324X
%https://www.sciencedirect.com/science/article/pii/B9780124172197000041